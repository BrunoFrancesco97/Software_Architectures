\documentclass[a4paper,11pt]{report}
\usepackage[latin1]{inputenc}
\usepackage[T1]{fontenc}
\usepackage[english]{babel}
\usepackage{fancyhdr}
\usepackage{graphicx}
\usepackage[hidelinks]{hyperref}
\usepackage{caption}
\usepackage{multirow}
\usepackage{array}
\lhead{}
\chead{}
\rhead{\fancyplain{}{\textit{\leftmark}}}

\title{
	\begin{figure}[!h]
		\vspace{-65mm}  
		\hspace{-30mm}
		\includegraphics[width=0.2\linewidth]{img/logo.jpg} \newline
		\hspace{-30mm}
		\label{fig:logo}
	\end{figure}
	\begin{flushleft}
		\vspace{-11mm}  
		\hspace{-30mm}
		\fontfamily{roboto}\selectfont Universit\'{a} \\
		\hspace{-30mm}
		Ca'Foscari\\
		\hspace{-30mm}
		Venezia\\
	\end{flushleft}
	\vspace{25mm}  
	\centerline{Master course}
	\vspace{5mm}
	\centerline{in Computer Science and Information Technology}
	\vspace{30mm}
	\centerline{Software Architectures}
	\vspace{5mm}
	\centerline{-}
}
\author{
	\LARGE \textbf{Task 1} \newline
} 
\date{
	\begin{flushleft}
		\vspace{30mm}
		\hspace*{-25mm}
		\textbf{Group 5} \newline
		\hspace*{-25mm}
		Bastianello Lorenzo, 874268\newline
		\hspace*{-25mm} 
		Bonomi Silvia, 867138\newline
		\hspace*{-25mm} 
		Bruno Francesco, 875812\newline
		\hspace*{-25mm} 
		Quaglia Beatrice Maria, 875332\newline
		\vspace{2mm} \newline
		\hspace*{-25mm}
		\textbf{Academic Year} \newline
		\hspace*{-25mm}
		2022 / 2023
	\end{flushleft}
}

\begin{document}
	\maketitle
	\makeatletter
	%\tableofcontents
	\chapter*{Introduction}
	\section*{Background}
	Submit a pdf file with your Katas about an hackerrank-style information system.
	\section*{Description}
	Il sistema proposto è una piattaforma hackerrank-style da adottare in un contento universitario/lavorativo in cui è richiesta la valutazione della formazione di una o più persone iscritte al servizio.
	\chapter*{Users}
	\section*{Number of users}
	1500 \textbf{studenti} contemporaneamente collegati previsti, più un numero minore di \textbf{professori} e \textbf{staff} la cui somma è pari a circa 50.\\
	20'000 utenti annui totali utilizzatori previsti all'interno dell'applicazione
	\section*{Type of users}
	\begin{itemize}
		\item \textbf{Professori}
		\item \textbf{Studenti}
		\item \textbf{Staff}
	\end{itemize}
	\chapter*{Requirements}
	\section*{Requirements}
	\begin{itemize}
		\item Suddivisione in aree di lavoro per ogni università/azienda;
		\item Ogni area di lavoro ha all'interno una suddivisione basata in corsi;
		\item Staff incaricato nella creazione di aree di lavoro e corsi, oltre che all'assistenza ordinaria, il tutto attraverso un pannello di controllo;
		\item Ogni corso all'interno di un'area di lavoro è amministrato da uno o più professori attraverso un pannello di controllo;
		\item Professori e studenti devono essere in grado di poter contattare lo staff per assistenza ordinaria;
		\item Ogni corso può contenere diverse sezioni in cui i professori possono adoperare:
		\begin{itemize}
			\item Sezione teorica;
			\item Sezione in cui sono presenti prove con valutazione (esami universitari o intervista);
			\item Sezione in cui un utente può misurare le proprie competenze tramite prove non valutate dai professori;
		\end{itemize}
	\item Una prova può essere a scelta multipla/domande aperte/input di file;
	\item Una prova può avere una scadenza;
	\item Ogni prova può essere ripresa in qualsiasi momento nel caso in cui essa non sia stata completata, sempre considerando la scadenza della stessa;
	\item Nel caso di test riguardanti l'ambito informatico, i linguaggi di programmazione supportati devono essere C, C++, Python, Java, SQL;
	\item Ogni prova viene controllata/eseguita dalla piattaforma in un ambiente protetto e separato dalla macchina su cui viene testato;
	\item Ogni professore all'interno del proprio corso può caricare dispense/materiale generico;
	\item Ogni professore all'interno del proprio corso gestito può creare uno o più prove;
	\item Ogni prova può disporre di un sistema anti-plagio;
	\item Ogni prova mette a disposizione un report finale visionabile dallo studente e dal professore, oltre che un report mostrante l'andamento totale del corso visionabile solo dal professore;
	\item Persistenza dei dati di utenti e prove tramite utilizzo di database;
	\item Ogni area di lavoro/corso/quiz ha un proprio link univoco condivisibile;
	\item Autenticazione tramite mail universitaria SSO???? Sappiamo farlo?
	\item Applicazione scalabile per un futuro ed ipotetico aumento di utenti presenti contemporaneamente e/o registrati;
	\item Applicazione deve resistere a picchi di utenti iscritti contemporaneamente;
	\item Applicazione quando necessario deve poter scalare le proprie risorse automaticamente a seconda della situazione;
	\item Applicazione costruita in moduli così da avere una più facile implementazione, integrazione e manutenzione degli stessi;
	\item Applicazione deve poter garantire il proprio funzionamento in qualsiasi istante della propria esecuzione;
	\item Applicazione deve garantire la presenza continua dei dati salvati, anche in caso di perdita di questi, tramite l'utilizzo di backup;
	\end{itemize}
	\chapter*{Additional Context}
	\begin{itemize}
		\item Availability is a first-class concern, also scalability;
		\item Il progetto deve avere un basso costo;
		\item Si prevede che nell'immediato futuro la piattaforma venga adottata da più società;
		\item user-experience is a must;
	\end{itemize}
\end{document}
